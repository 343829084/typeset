\chapter{插图}

尽量用矢量图,少用点阵图(屏幕截图只适合用于step-by-step介绍软件操作)。

\section{绘图软件}

\mybooktitle 的排版过程中我使用了多种绘图工具,这里简单列举一下。
这里举的插图例子大多数可以在样章 \footnote{\myurl{http://vdisk.weibo.com/s/mtupb}} 第6章中找到。
以下这些软件都可以生成矢量图(EPS或PDF文件),印刷效果较好。

\subsection{Graphviz}
适合绘制依赖关系图,包括 \fn{\#include} 头文件的关系图(书p.132图~6-1)、
简单的 class 继承关系图等。

优点:自动布局。

缺点:如果对自动布局不满意,不易手工调整。
\subsection{gpic}

适合绘制线框图,例如数据结构。本文的图~\ref{fig:sharedptr} 就是用 gpic 绘制的。
gpic 可参考 Brian W. Kernighan 的 《PIC---A Graphics Language for Typesetting, User Manual》
和 Eric S. Raymond 的《Making Pictures with GNU PIC》。

缺点:无法输入中文。

\subsection{METAPOST}
适合绘制函数图像,例如书p.349图~9-5。

\subsection{Visio}
适合绘制较复杂的 class diagram(书p.132图~6-2),sequence diagram(例如书p.170图~6-12)。
需要安装 	
Visio Stencil and Template for UML 2.2
\footnote{http://www.softwarestencils.com/uml/index.html}。

可以输入中文,可以可视化编辑,可以导出为 PDF 文件,经过剪裁去掉白边
\footnote{用 \fn{pdfcrop} 工具 \S \ref{sec:pdfcrop} 按内容剪裁。}
之后可以方便地嵌入 \LaTeX 文档。

缺点:多幅内容相近的图片不易统一修改,例如《为什么多线程读写 \fn{shared_ptr} 要加锁?》%
\footnote{\myurl{http://chenshuo.googlecode.com/files/CppEngineering.pdf}}
中绘制的多幅 \fn{shared_ptr} 内部指针变化图。

\subsection{Word}
适合复杂的表格,例如书p.161表~6-1。
可以输出为PDF文件,剪裁之后以插图的方式嵌入 \LaTeX 文档。

\subsection{Excel}
适合绘制点数较少的性能数据图($x$-$y$ 坐标系),例如书p.147图~6-3。

\subsection{Gnuplot}
适合绘制点数较多的性能数据图,例如书p.150图~6-6。

%\section{图片工具}
%\subsection{pdfcrop}
%\subsection{ps2eps}
%\subsection{eps2png}

\section{待续……}
