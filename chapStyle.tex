\chapter{样式}

\section{转义字符}
注意“\#”是 \TeX 元字符,因此 C\# 要写作 \fn{C\bs\#}。
类似的“\textasciitilde”也是元字符,
要写作\mn{textasciitilde}。
\index{命令!textasciitilde@\mn{textasciitilde}}
这两个字符在URL中也经常出现,要特别小心。

C/C++代码的标识符中经常出现下划线“_”,例如 \fn{boost::shared_ptr}、
\fn{random_\linebreak[0]shuffle} 等。
为了避免每次都转义,可以使用 \fn{underscore} 宏包,
将下划线变为普通字符。
\index{宏包!underscore@\fn{underscore}}
注意这里 \fn{random_shuffle} 在行尾断字,
那么下划线之后不应该出现连字号“-”,
因此应写为 \verb|random_\linebreak[0]shuffle|。

\section{列表}
\fn{enumitem}
\index{宏包!enumitem@\fn{enumitem}}

\section{章节标题}
章节标题无标点,因此 \S \ref{sec:whyTypesetting} 是错的。

\subsection{编号}
单个小节不编号,因此 \S \ref{subsec:noChineseItalic} 和本小节是错的。

\section{图表编号}

\section{脚注}
\subsection{编号}
\begin{Code}
\@addtoreset{footnote}{page}
\end{Code}
无效,必须使用 \fn{footmisc} 宏包的 \fn{perpage} 选项,
例如 \sfn{\bs usepackage[perpage]\{footmisc\}}。
\index{宏包!footmisc@\fn{footmisc}}
\subsection{置底}


\section{参考文献}
技术书籍不是学术著作,可不必使用 BibTeX 工具,直接排版参考文献即可。